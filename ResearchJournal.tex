% Please do not change the document class
\documentclass{scrartcl}

% Please do not change these packages
\usepackage[hidelinks]{hyperref}
\usepackage[none]{hyphenat}
\usepackage{setspace}
\doublespace

% You may add additional packages here
\usepackage{amsmath}

\title{Research Journal}

\subtitle{COMP130 - Game Architecture}

\author{1608351}

\begin{document}

\maketitle

\section{Introduction}

As one of the five core STEM subjects, computer science degrees have a distinct bias towards male students, with only 25 percent of STEM graduates in 2016, being female \cite{12}. Surprisingly the number of women studying computer science in the UK, has been decreasing since the 1980's \textit{`female representation in computing education has been in decline for the past 30 years"} (\cite{gender}, 2015, p.1). Yet in Malaysia, most programmers are female \cite{8}, suggesting that the UK's unbalanced gender ratio, may be influenced by our perception of computer science and of women. This paper aims to explore the existing literature on gender diversity within computer science education, seeking probable causes and potential solutions to the issue.

\section{Encouraging Gender Diversity in Computer Science}

There remains a strong lack of female students on computer science degree programs \cite{3}. This is likely attributed to enduring gender stereotypes, discouraging women from an early age to engage their interests in computer science related areas; \textit{`Early positive exposure to the computer sciences appears to be the first step in mending the gender gap and encouraging more young girls and women to explore careers in computer sciences"}(\cite{3}, 2015, p.5, 6). Choosing to enter such a male-dominated workplace as an adult is a further deterrent, especially with gender bias at school often resulting in women feeling less confident in skills beneficial to computer science, such as problem-solving. Therefore, improving the culture of gender bias at school may help encourage more women to the field in the future. 

Perceptions of computer science degrees and related careers, are another likely contributor to the lack of women in the industry. Women tend to have stronger social inclinations then men, and so the common perception of computer science students as \textit{` intelligent but deficient in interpersonal skills"} (\cite{7}, p.1), may be a further discouragement. Programming and computer science in general, is often considered a socially-isolated activity \cite{5}, however, in the gaming industry especially, a strong focus on teamwork and the use of agile principles encourages a much more social atmosphere \cite{4}. Agile is an adaptive methodology for software development, and is becoming increasingly popular in the gaming industry. Yet agile principles are often not introduced at secondary, nor higher-level, education \cite{2}. Pair programming, an important agile method, has been found particularly effective in teaching programming in the classroom, \textit{`Pair programming is the obvious winner, bringing clear advantages in motivation and “classroom mood”, code quality and, depending on pair composition, grades."} (\cite{2}, 2016, p.301), and being in a supportive pair has been found to improve programming confidence amongst female students \cite{3}. Introducing agile principles in computer science programs at schools, may help diminish the social stereotypes attributed to the field and encourage more students, including females, to pursue related careers.

There also exists a perception of women having a lower aptitude for computer science then men, and female students studying computer science generally report lower levels of confidence in their skill then their male peers \cite{7}. This is despite several pioneers for computer science, such as Ada Lovelace and Edith Clarke, being female. Women entering onto computer science degree programs, often have less experience and tend to have developed an interest in programming later in life, when compared to their male peers \cite{9}. 

Interestingly, most programmers in Malaysia are female \cite{8}. In Malaysia, computer science is generally not considered to be a difficult subject to master, suggesting the lack of female programmers in Europe and the USA may be predominantly due to the general perception of computer science in such countries \textit{`We conclude that young Malaysians have a different perception of CS/IT compared to the Western world"} (\cite{8}, 2006 ,114).

Studies have found a person's gender can affect their learning-style, with female students preferring to have an idea of the real-world application of tasks assigned to them in order to develop interest \textit{`Females want an overview and a connection to reality when learning a new computer language, while men are able to concentrate only on the subject they need to learn without the same need for reality connection or overview."} (\cite{10}, 2015, p.26). This suggests that computer science programs may benefit from considering gender-differences when designing curriculums, to encourage and maintain a more balanced gender ratio.

Ariel Schlesinger has suggested the need for a more female based programming language \cite{11}. However, Schlesinger has not proposed a functioning language or published a complete study, and in opposition to her proposition, evidence suggests no need for programming language targeted towards women, instead changing the way programming languages are taught will have a greater benefit in encouraging more women to the subject \cite{10}. 

A possible solution to the gender ratio, is to provide meaningful assignments, which have real-world application and are beneficial to society. Sax et al \cite{15} noted women found greater value in supporting others and contributing to social change then men, and suggests the abstract nature of computer science education may be a notable cause of the current gender ratio. Introducing assignments which have a clear impact on the local community, may increase engagement of female students, who tend to be more socially inclined and prefer tasks relevant to the real-world \cite{14}.

\section{Improving Computer Science Education}

Improving the perception, structure and definition of computer science education in general, may not only encourage more female students but also improve the quality of graduates. Stroustrup suggests a \textit{`disconnect between computer science education and what industry needs."}\cite{1}, with industry professionals claiming graduates are lacking in programming proficiency. In addition, software engineering practices, such as the use of ontology, user stories and timeboxing tend to be undervalued by students \cite{2} \cite{6}. This may be due to such tasks being imposed on students, rather than being self-directed, as would be the case in industry. Introducing the use of such practices during the development of a group software project may help students recognise their value. 

Computer science education, Stroustrup argues, needs a standardised structure with a clearer definition of what an individual requires to earn the title of Computer Scientist, perhaps leading to the licensing of Computer Science professionals. To achieve such license, professionals should not purely possess a generalised understanding of computer science, but have obtained a specialism within the field \cite{1}. A clear explanation of the role of a Computer Scientist for young people and a stronger emphasis on computer science from primary education onwards, may help encourage a more diverse range of students. 

\section{Conclusions}

In conclusion, the literature review suggests programming languages do not need to be adapted for women. Instead understanding and accommodating the differences in learning styles between genders, may be beneficial in encouraging diversity in computer science education \cite{10}. For women, this means connecting tasks to real-life examples and providing a stronger overview of how the various aspects of computer science, from CPU functions to coding, connect. Designing courses with a stronger emphasis on projects which may benefit the local community, may not only increase the interest of female students but also provide more exposure of computer science in general. Encouraging girls from an early age to practice and enjoy programming, providing examples of female computing professionals and diminishing the current perception of computer science as a socially-isolating and masculine activity, may lead to more women considering computer science degrees in the future.



\bibliographystyle{ieeetran}
\bibliography{journalreferences}

\end{document}